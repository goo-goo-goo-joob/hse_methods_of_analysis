%!TEX TS-program = xelatex

\documentclass[a4paper,14pt]{article}

%%% Работа с русским языком
\usepackage[english,russian]{babel}   %% загружает пакет многоязыковой вёрстки
\usepackage{fontspec}      %% подготавливает загрузку шрифтов Open Type, True Type и др.
\defaultfontfeatures{Ligatures={TeX},Renderer=Basic}  %% свойства шрифтов по умолчанию
\setmainfont[Ligatures={TeX,Historic}]{Calibri} %% задаёт основной шрифт документа
\setsansfont{Comic Sans MS}                    %% задаёт шрифт без засечек
\setmonofont{Courier New}
\usepackage{indentfirst}
\frenchspacing

\renewcommand{\epsilon}{\ensuremath{\varepsilon}}
\renewcommand{\phi}{\ensuremath{\varphi}}
\renewcommand{\kappa}{\ensuremath{\varkappa}}
\renewcommand{\le}{\ensuremath{\leqslant}}
\renewcommand{\leq}{\ensuremath{\leqslant}}
\renewcommand{\ge}{\ensuremath{\geqslant}}
\renewcommand{\geq}{\ensuremath{\geqslant}}
\renewcommand{\emptyset}{\varnothing}

%%% Дополнительная работа с математикой
\usepackage{amsmath,amsfonts,amssymb,amsthm,mathtools} % AMS
\usepackage{icomma} % "Умная" запятая: $0,2$ --- число, $0, 2$ --- перечисление

%% Номера формул
%\mathtoolsset{showonlyrefs=true} % Показывать номера только у тех формул, на которые есть \eqref{} в тексте.
%\usepackage{leqno} % Нумерация формул слева	

%% Перенос знаков в формулах (по Львовскому)
\newcommand*{\hm}[1]{#1\nobreak\discretionary{}
	{\hbox{$\mathsurround=0pt #1$}}{}}

%%% Работа с картинками
\usepackage{graphicx}  % Для вставки рисунков
\graphicspath{{images/}}  % папки с картинками
\setlength\fboxsep{3pt} % Отступ рамки \fbox{} от рисунка
\setlength\fboxrule{1pt} % Толщина линий рамки \fbox{}
\usepackage{wrapfig} % Обтекание рисунков текстом

%%% Работа с таблицами
\usepackage{array,tabularx,tabulary,booktabs} % Дополнительная работа с таблицами
\usepackage{longtable}  % Длинные таблицы
\usepackage{multirow} % Слияние строк в таблице
\usepackage{float}% http://ctan.org/pkg/float

%%% Программирование
\usepackage{etoolbox} % логические операторы


%%% Страница
\usepackage{extsizes} % Возможность сделать 14-й шрифт
\usepackage{geometry} % Простой способ задавать поля
\geometry{top=20mm}
\geometry{bottom=20mm}
\geometry{left=30mm}
\geometry{right=15mm}
%
%\usepackage{fancyhdr} % Колонтитулы
% 	\pagestyle{fancy}
%\renewcommand{\headrulewidth}{0pt}  % Толщина линейки, отчеркивающей верхний колонтитул
% 	\lfoot{Нижний левый}
% 	\rfoot{Нижний правый}
% 	\rhead{Верхний правый}
% 	\chead{Верхний в центре}
% 	\lhead{Верхний левый}
%	\cfoot{Нижний в центре} % По умолчанию здесь номер страницы

\usepackage{setspace} % Интерлиньяж
\onehalfspacing % Интерлиньяж 1.5
%\doublespacing % Интерлиньяж 2
%\singlespacing % Интерлиньяж 1

\usepackage{lastpage} % Узнать, сколько всего страниц в документе.

\usepackage{soul} % Модификаторы начертания

\usepackage{hyperref}
\usepackage[usenames,dvipsnames,svgnames,table,rgb]{xcolor}
\hypersetup{				% Гиперссылки
	unicode=true,           % русские буквы в раздела PDF
	pdftitle={Отчет о прохождении практики},   % Заголовок
	pdfauthor={Самоделкина М.В.},      % Автор
	pdfsubject={Отчет о прохождении практики},      % Тема
	pdfcreator={Самоделкина М.В.}, % Создатель
	pdfproducer={Самоделкина М.В.}, % Производитель
	pdfkeywords={keyword1} {key2} {key3}, % Ключевые слова
	colorlinks=true,       	% false: ссылки в рамках; true: цветные ссылки
	linkcolor=blue,          % внутренние ссылки
	citecolor=black,        % на библиографию
	filecolor=magenta,      % на файлы
	urlcolor=blue           % на URL
}
\makeatletter 
\def\@biblabel#1{#1. } 
\makeatother
\usepackage{cite} % Работа с библиографией
%\usepackage[superscript]{cite} % Ссылки в верхних индексах
%\usepackage[nocompress]{cite} % 
\usepackage{csquotes} % Еще инструменты для ссылок

\usepackage{multicol} % Несколько колонок

\usepackage{tikz} % Работа с графикой
\usepackage{pgfplots}
\usepackage{pgfplotstable}

% ГОСТ заголовки
\usepackage[font=small]{caption}
%\captionsetup[table]{justification=centering, labelsep = newline} % Таблицы по правобу краю
%\captionsetup[figure]{justification=centering} % Картинки по центру
\usepackage{ dsfont }

\newcommand{\tablecaption}[1]{\addtocounter{table}{1}\small \begin{flushright}\tablename \ \thetable\end{flushright}%	
\begin{center}#1\end{center}}

\newcommand{\imref}[1]{рис.~\ref{#1}}

\usepackage{multirow}
\usepackage{spreadtab}
\newcolumntype{K}[1]{@{}>{\centering\arraybackslash}p{#1cm}@{}}


\usepackage{xparse}
\usepackage{fancyvrb}

\RecustomVerbatimCommand{\VerbatimInput}{VerbatimInput}
{
	fontsize=\footnotesize    
}

\newcolumntype{?}[1]{!{\vrule width #1}}

\usepackage{tocloft}
\renewcommand{\cftsecleader}{\cftdotfill{\cftdotsep}}

\usepackage{pdfpages}

\usepackage{longtable}
\begin{document} % конец преамбулы, начало документа
\begin{titlepage}
	\begin{center}
		ПРАВИТЕЛЬСТВО РОССИЙСКОЙ ФЕДЕРАЦИИ \\
 		ФЕДЕРАЛЬНОЕ  ГОСУДАРСТВЕННОЕ АВТОНОМНОЕ \\
		ОБРАЗОВАТЕЛЬНОЕ УЧРЕЖДЕНИЕ ВЫСШЕГО ОБРАЗОВАНИЯ\\
		«НАЦИОНАЛЬНЫЙ ИССЛЕДОВАТЕЛЬСКИЙ УНИВЕРСИТЕТ\\
		«ВЫСШАЯ ШКОЛА ЭКОНОМИКИ»
	\end{center}
	
	\begin{center}
		\textbf{Московский институт электроники и математики}
		
		\textbf{Им. А.Н.Тихонова НИУ ВШЭ}
		
		\vspace{2ex}
		
		\textbf{Направление 01.03.04. Прикладная математика \\
			Бакалаврская программа <<Прикладная математика>>}
	\end{center}
	\vspace{1ex}	
	
	\vspace{1ex}
	\begin{center}
		\textbf{Отчет по самостоятельной работе \\
			по дисциплине <<Методы анализа стохастических взаимосвязей>>\\
			часть 1
	}
	\end{center}	

	\vspace{2ex}
	\vfill
	
	\vspace{2ex}
	
	\begin{flushright}
		\textbf{Бригада №7:}
		
		\vspace{2ex}
		
		Ремизова Анна Петровна, 4 курс, БПМ174
		
		Самоделкина Мария Владимировна, 4 курс, БПМ174

	\end{flushright}

	\vspace{5ex}
	\begin{center}
		Москва \the\year \, г.
	\end{center}
	
\end{titlepage}
\addtocounter{page}{1}
%\title{Отчет о прохождении производственной практики}
%\date{}
%\maketitle
%\includepdf{data/title2.pdf}
\tableofcontents
\pagebreak

\section{Общая постановка задачи}
\subsection{Формулировка прикладной проблемы}
Определение размера индивидуальных взносов медицинского страхования по информации о человеке: полу, возрасту, индексу массы тела, количеству детей, наличию вредных привычек и региону.

\subsection{Потенциальные потребители решения. Задачи, которые они смогут решать,	используя полученные результаты}
Полученные решения могут быть полезны медицинским страховым компаниям. Преимущественно это компании из США, поскольку медицинское обслуживание в этой стране имеет страховой характер. Набор данных соответствует северным районам США - в них находятся мегалополисы с большой численностью населения

Используя полученные результаты, потребители смогут оценивать размер индивидуальных взносов, основываясь на индивидуальных характеристиках человека.

\subsection{Основные гипотезы, которые планируется проверить в рамках решения задачи}
В таблице \ref{tab:table1} представлены индивидуальные характеристики человека, используемые для анализы медицинских взносов. 

\begin{table}[H]
\begin{center}
	\begin{tabular}{ | l | p{4cm} | p{2.6cm} | p{4.2cm} | p{3.7cm} |}
		\hline
		№ & Характеристика объекта & Название переменной & Шкала измерения & Роль переменной \\ \hline
		1 & Возраст & age & Относительная & Независимая \\ \hline
		2 & Пол & sex & Номинальная (дихотомическая) & Независимая \\ \hline
		3 & Индекс массы тела & bmi & Относительная & Независимая \\ \hline
		4 & Число детей & children & Относительная & Независимая \\ \hline
		5 & Наличие вредных привычек (курение) & smoker & Номинальная (дихотомическая) & Независимая \\ \hline
		6 & Регион & region & Номинальная (дихотомическая) & Независимая \\ \hline
		7 & Индивидуальные взносы & charges & Относительная & Зависимая \\ \hline
	\end{tabular}
\end{center}
\caption{Описание факторов, учтенных в анализе.}
\label{tab:table1}
\end{table}
Сформулируем гипотезы о статистической взаимосвязи зависимых и независимых переменных:
\begin{enumerate}
	\item Индекс массы тела человека до некоторого возраста растет, а затем уменьшается.
	
	\item Цена медицинской страховки уменьшается с увеличением индекса массы тела до 18.5, затем размер взносов практически не меняется в зависимости от индекса массы тела в промежутке от 18.5 до 25, а затем с увеличением индекса снова растет.
	
	\item С возрастом размер медицинских страховых взносов растет. При этом скорость роста взносов у мужчин с возрастом выше, чем у женщин.
\end{enumerate}

\subsection{Основные источники данных}
Данные были взяты из \href{https://github.com/stedy/Machine-Learning-with-R-datasets/blob/master/insurance.csv}{репозитория в GitHub}. Интервал нормы индекса массы тела здорового человека нашли на  \href{https://www.cdc.gov/healthyweight/assessing/bmi/adult_bmi/index.html}{сайте департамента здоровья в США}.
	

\end{document} % конец документа