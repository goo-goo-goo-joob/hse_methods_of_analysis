%!TEX TS-program = xelatex

\documentclass[a4paper,14pt]{article}

\input{data/preambular.tex}
\begin{document} % конец преамбулы, начало документа
\input{data/title.tex}
%\title{Отчет о прохождении производственной практики}
%\date{}
%\maketitle
%\includepdf{data/title2.pdf}
\tableofcontents
\pagebreak

\section{Общая постановка задачи}
\subsection{Формулировка прикладной проблемы}
Определение размера минимального индивидуального взноса у некурящих женщин в возрасте до 22 лет без детей с индексом массы тела в диапазоне от 18.5 до 25.
%Определение размера индивидуальных взносов медицинского страхования по информации о человеке: полу, возрасту, индексу массы тела, количеству детей, наличию вредных привычек и региону.

\subsection{Потенциальные потребители решения. Задачи, которые они смогут решать,	используя полученные результаты}
Полученные решения могут быть полезны медицинским страховым компаниям. Преимущественно это компании из США, поскольку медицинское обслуживание в этой стране имеет страховой характер. Набор данных соответствует северным районам США - в них находятся мегалополисы с большой численностью населения.

Используя полученные результаты, потребители смогут оценивать размер индивидуальных взносов, основываясь на индивидуальных характеристиках человека.

\subsection{Основные гипотезы, которые планируется проверить в рамках решения задачи}
В таблице \ref{tab:table1} представлены индивидуальные характеристики человека, используемые для анализа медицинских взносов. 

\begin{table}[H]
\begin{center}
	\begin{tabular}{ | l | p{4cm} | p{2.6cm} | p{4.2cm} | p{3.7cm} |}
		\hline
		№ & Характеристика объекта & Название переменной & Шкала измерения & Роль переменной \\ \hline
		1 & Возраст & age & Относительная & Независимая \\ \hline
		2 & Пол & sex & Номинальная (дихотомическая) & Независимая \\ \hline
		3 & Индекс массы тела & bmi & Относительная & Независимая \\ \hline
		4 & Число детей & children & Относительная & Независимая \\ \hline
		5 & Наличие вредных привычек (курение) & smoker & Номинальная (дихотомическая) & Независимая \\ \hline
		6 & Регион & region & Номинальная (дихотомическая) & Независимая \\ \hline
		7 & Индивидуальные взносы & charges & Относительная & Зависимая \\ \hline
	\end{tabular}
\end{center}
\caption{Описание факторов, учтенных в анализе.}
\label{tab:table1}
\end{table}
Сформулируем гипотезы о статистической взаимосвязи зависимых и независимых переменных:
\begin{enumerate}
	\item У людей с вредными привычками размер страховых взносов больше, чем у людей без вредных привычек.
	%\item Индекс массы тела человека до некоторого возраста растет, а затем уменьшается.
	
	\item Цена медицинской страховки уменьшается с увеличением индекса массы тела до 18.5, затем размер взносов практически не меняется в зависимости от индекса массы тела в промежутке от 18.5 до 25, а затем с увеличением индекса снова растет.
	
	\item С возрастом размер медицинских страховых взносов растет. При этом скорость роста взносов у мужчин с возрастом выше, чем у женщин.
\end{enumerate}

\subsection{Основные источники данных}
Данные были взяты из \href{https://github.com/stedy/Machine-Learning-with-R-datasets/blob/master/insurance.csv}{репозитория в GitHub}. Интервал нормы индекса массы тела здорового человека нашли на  \href{https://www.cdc.gov/healthyweight/assessing/bmi/adult_bmi/index.html}{сайте департамента здоровья США}.
	

\end{document} % конец документа